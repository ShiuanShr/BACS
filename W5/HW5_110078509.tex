% Options for packages loaded elsewhere
\PassOptionsToPackage{unicode}{hyperref}
\PassOptionsToPackage{hyphens}{url}
%
\documentclass[
]{article}
\usepackage{amsmath,amssymb}
\usepackage{lmodern}
\usepackage{ifxetex,ifluatex}
\ifnum 0\ifxetex 1\fi\ifluatex 1\fi=0 % if pdftex
  \usepackage[T1]{fontenc}
  \usepackage[utf8]{inputenc}
  \usepackage{textcomp} % provide euro and other symbols
\else % if luatex or xetex
  \usepackage{unicode-math}
  \defaultfontfeatures{Scale=MatchLowercase}
  \defaultfontfeatures[\rmfamily]{Ligatures=TeX,Scale=1}
\fi
% Use upquote if available, for straight quotes in verbatim environments
\IfFileExists{upquote.sty}{\usepackage{upquote}}{}
\IfFileExists{microtype.sty}{% use microtype if available
  \usepackage[]{microtype}
  \UseMicrotypeSet[protrusion]{basicmath} % disable protrusion for tt fonts
}{}
\makeatletter
\@ifundefined{KOMAClassName}{% if non-KOMA class
  \IfFileExists{parskip.sty}{%
    \usepackage{parskip}
  }{% else
    \setlength{\parindent}{0pt}
    \setlength{\parskip}{6pt plus 2pt minus 1pt}}
}{% if KOMA class
  \KOMAoptions{parskip=half}}
\makeatother
\usepackage{xcolor}
\IfFileExists{xurl.sty}{\usepackage{xurl}}{} % add URL line breaks if available
\IfFileExists{bookmark.sty}{\usepackage{bookmark}}{\usepackage{hyperref}}
\hypersetup{
  pdftitle={HW},
  pdfauthor={110078509},
  hidelinks,
  pdfcreator={LaTeX via pandoc}}
\urlstyle{same} % disable monospaced font for URLs
\usepackage[margin=1in]{geometry}
\usepackage{color}
\usepackage{fancyvrb}
\newcommand{\VerbBar}{|}
\newcommand{\VERB}{\Verb[commandchars=\\\{\}]}
\DefineVerbatimEnvironment{Highlighting}{Verbatim}{commandchars=\\\{\}}
% Add ',fontsize=\small' for more characters per line
\usepackage{framed}
\definecolor{shadecolor}{RGB}{248,248,248}
\newenvironment{Shaded}{\begin{snugshade}}{\end{snugshade}}
\newcommand{\AlertTok}[1]{\textcolor[rgb]{0.94,0.16,0.16}{#1}}
\newcommand{\AnnotationTok}[1]{\textcolor[rgb]{0.56,0.35,0.01}{\textbf{\textit{#1}}}}
\newcommand{\AttributeTok}[1]{\textcolor[rgb]{0.77,0.63,0.00}{#1}}
\newcommand{\BaseNTok}[1]{\textcolor[rgb]{0.00,0.00,0.81}{#1}}
\newcommand{\BuiltInTok}[1]{#1}
\newcommand{\CharTok}[1]{\textcolor[rgb]{0.31,0.60,0.02}{#1}}
\newcommand{\CommentTok}[1]{\textcolor[rgb]{0.56,0.35,0.01}{\textit{#1}}}
\newcommand{\CommentVarTok}[1]{\textcolor[rgb]{0.56,0.35,0.01}{\textbf{\textit{#1}}}}
\newcommand{\ConstantTok}[1]{\textcolor[rgb]{0.00,0.00,0.00}{#1}}
\newcommand{\ControlFlowTok}[1]{\textcolor[rgb]{0.13,0.29,0.53}{\textbf{#1}}}
\newcommand{\DataTypeTok}[1]{\textcolor[rgb]{0.13,0.29,0.53}{#1}}
\newcommand{\DecValTok}[1]{\textcolor[rgb]{0.00,0.00,0.81}{#1}}
\newcommand{\DocumentationTok}[1]{\textcolor[rgb]{0.56,0.35,0.01}{\textbf{\textit{#1}}}}
\newcommand{\ErrorTok}[1]{\textcolor[rgb]{0.64,0.00,0.00}{\textbf{#1}}}
\newcommand{\ExtensionTok}[1]{#1}
\newcommand{\FloatTok}[1]{\textcolor[rgb]{0.00,0.00,0.81}{#1}}
\newcommand{\FunctionTok}[1]{\textcolor[rgb]{0.00,0.00,0.00}{#1}}
\newcommand{\ImportTok}[1]{#1}
\newcommand{\InformationTok}[1]{\textcolor[rgb]{0.56,0.35,0.01}{\textbf{\textit{#1}}}}
\newcommand{\KeywordTok}[1]{\textcolor[rgb]{0.13,0.29,0.53}{\textbf{#1}}}
\newcommand{\NormalTok}[1]{#1}
\newcommand{\OperatorTok}[1]{\textcolor[rgb]{0.81,0.36,0.00}{\textbf{#1}}}
\newcommand{\OtherTok}[1]{\textcolor[rgb]{0.56,0.35,0.01}{#1}}
\newcommand{\PreprocessorTok}[1]{\textcolor[rgb]{0.56,0.35,0.01}{\textit{#1}}}
\newcommand{\RegionMarkerTok}[1]{#1}
\newcommand{\SpecialCharTok}[1]{\textcolor[rgb]{0.00,0.00,0.00}{#1}}
\newcommand{\SpecialStringTok}[1]{\textcolor[rgb]{0.31,0.60,0.02}{#1}}
\newcommand{\StringTok}[1]{\textcolor[rgb]{0.31,0.60,0.02}{#1}}
\newcommand{\VariableTok}[1]{\textcolor[rgb]{0.00,0.00,0.00}{#1}}
\newcommand{\VerbatimStringTok}[1]{\textcolor[rgb]{0.31,0.60,0.02}{#1}}
\newcommand{\WarningTok}[1]{\textcolor[rgb]{0.56,0.35,0.01}{\textbf{\textit{#1}}}}
\usepackage{longtable,booktabs,array}
\usepackage{calc} % for calculating minipage widths
% Correct order of tables after \paragraph or \subparagraph
\usepackage{etoolbox}
\makeatletter
\patchcmd\longtable{\par}{\if@noskipsec\mbox{}\fi\par}{}{}
\makeatother
% Allow footnotes in longtable head/foot
\IfFileExists{footnotehyper.sty}{\usepackage{footnotehyper}}{\usepackage{footnote}}
\makesavenoteenv{longtable}
\usepackage{graphicx}
\makeatletter
\def\maxwidth{\ifdim\Gin@nat@width>\linewidth\linewidth\else\Gin@nat@width\fi}
\def\maxheight{\ifdim\Gin@nat@height>\textheight\textheight\else\Gin@nat@height\fi}
\makeatother
% Scale images if necessary, so that they will not overflow the page
% margins by default, and it is still possible to overwrite the defaults
% using explicit options in \includegraphics[width, height, ...]{}
\setkeys{Gin}{width=\maxwidth,height=\maxheight,keepaspectratio}
% Set default figure placement to htbp
\makeatletter
\def\fps@figure{htbp}
\makeatother
\setlength{\emergencystretch}{3em} % prevent overfull lines
\providecommand{\tightlist}{%
  \setlength{\itemsep}{0pt}\setlength{\parskip}{0pt}}
\setcounter{secnumdepth}{-\maxdimen} % remove section numbering
\ifluatex
  \usepackage{selnolig}  % disable illegal ligatures
\fi

\title{HW}
\author{110078509}
\date{20220319}

\begin{document}
\maketitle

\begin{center}\rule{0.5\linewidth}{0.5pt}\end{center}

\hypertarget{question-1}{%
\subsubsection{Question 1}\label{question-1}}

\begin{longtable}[]{@{}l@{}}
\toprule
\endhead
i. Would this scenario create systematic or random error (or both or
neither)? \\
ii. Which part of the t-statistic or significance (diff, sd, n, alpha)
would be affected? \\
iii. Will it increase or decrease our power to reject the null
hypothesis? \\
iv. Which kind of error (Type I or Type II) becomes more likely because
of this scenario? \\
\bottomrule
\end{longtable}

\hypertarget{a.}{%
\paragraph{a.}\label{a.}}

You discover that your colleague wanted to target the general population
of Taiwanese users of the product. However, he only collected data from
a pool of young consumers, and missed many older customers who you
suspect might use the product much less every day.

\hypertarget{i.-systematic-error}{%
\subsection{\texorpdfstring{\#\#\#\#\# i. \emph{systematic
error}}{\#\#\#\#\# i. systematic error}}\label{i.-systematic-error}}

\hypertarget{ii.-sd-diff-n-are-affected.}{%
\subparagraph{\texorpdfstring{ii. \emph{sd, diff, n are
affected.}}{ii. sd, diff, n are affected.}}\label{ii.-sd-diff-n-are-affected.}}

\emph{Explains:}

Because we missed many older customers. Therefore, the alternative
distribution we obtained is too narrow compared to the real one. Hence,
the sd is affected.

And the sample size we got is fewer than it shall be if we want to
maintain a randomized selected sample assumption. Hence, n is affected.

The diff is ( x mean -μο), in this scenario, the x means is not accurate
anymore, because we missed plenty of samples lower than the average.
Hence the diff is affected also.

\begin{longtable}[]{@{}
  >{\raggedright\arraybackslash}p{(\columnwidth - 0\tabcolsep) * \real{1.00}}@{}}
\toprule
\endhead
\#\#\#\#\# iii. It decreases our power to reject the null hypothesis. \\
\emph{Explains:} \\
Manipulate the sliding window to figure out. \\
\bottomrule
\end{longtable}

\hypertarget{iv.-typeii-error}{%
\subparagraph{iv. TypeII error}\label{iv.-typeii-error}}

\emph{Explains:}

It also called false negative. The data of elder people should be
considered in, however, they are not.Therefore, it Type II error.
------------------------------------------------------------------------------------------------

\hypertarget{b.you-find-that-20-of-the-respondents-are-reporting-data-from-the-wrong-wearable-device-so-they-should-be-removed-from-the-data.-these-20-people-are-just-like-the-others-in-every-other-respect.}{%
\paragraph{b.You find that 20 of the respondents are reporting data from
the wrong wearable device, so they should be removed from the data.
These 20 people are just like the others in every other
respect.}\label{b.you-find-that-20-of-the-respondents-are-reporting-data-from-the-wrong-wearable-device-so-they-should-be-removed-from-the-data.-these-20-people-are-just-like-the-others-in-every-other-respect.}}

\hypertarget{i.-random-error}{%
\subsection{\texorpdfstring{\#\#\#\#\# i. \emph{random
error}}{\#\#\#\#\# i. random error}}\label{i.-random-error}}

\hypertarget{ii.-n-would-be-lower.}{%
\subparagraph{\texorpdfstring{ii. \emph{n would be lower.
}}{ii. n would be lower. }}\label{ii.-n-would-be-lower.}}

\emph{Explains:}

\hypertarget{because-these-noisy-data-shall-not-be-considered-in.}{%
\subsection{Because these noisy data shall not be considered
in.}\label{because-these-noisy-data-shall-not-be-considered-in.}}

\hypertarget{iii.-it-decrease-the-power-to-reject-the-null-hypothesis}{%
\subparagraph{\texorpdfstring{iii. \emph{It decrease the power to reject
the null
hypothesis}}{iii. It decrease the power to reject the null hypothesis}}\label{iii.-it-decrease-the-power-to-reject-the-null-hypothesis}}

\emph{Reason:}

\hypertarget{manipulate-the-sliding-window-to-figure-out.}{%
\subsection{Manipulate the sliding window to figure
out.}\label{manipulate-the-sliding-window-to-figure-out.}}

\hypertarget{iv.-type-i-error}{%
\subparagraph{\texorpdfstring{iv. \emph{Type I
Error}}{iv. Type I Error}}\label{iv.-type-i-error}}

\emph{Explains:}

\hypertarget{it-also-called-false-positive.-in-this-scenario-we-considered-the-error-into-our-analysis-however-we-shall-not.-therefore-it-false-positive.}{%
\subsection{It also called false positive. In this scenario, we
considered the error into our analysis, however, we shall not.
Therefore, it false
positive.}\label{it-also-called-false-positive.-in-this-scenario-we-considered-the-error-into-our-analysis-however-we-shall-not.-therefore-it-false-positive.}}

\hypertarget{c.-a-very-annoying-professor-visiting-your-company-has-criticized-your-colleagues-95-confidence-criteria-and-has-suggested-relaxing-it-to-just-90.}{%
\paragraph{c.~A very annoying professor visiting your company has
criticized your colleague's ``95\% confidence'' criteria, and has
suggested relaxing it to just
90\%.}\label{c.-a-very-annoying-professor-visiting-your-company-has-criticized-your-colleagues-95-confidence-criteria-and-has-suggested-relaxing-it-to-just-90.}}

\begin{center}\rule{0.5\linewidth}{0.5pt}\end{center}

\hypertarget{i.-neither.}{%
\subparagraph{\texorpdfstring{i.
\emph{neither.}}{i. neither.}}\label{i.-neither.}}

\emph{Explains:}

\hypertarget{it-just-the-change-of-the-confidence-level-based-on-the-professors-suggestion.}{%
\subsection{It just the change of the confidence level based on the
professor's
suggestion.}\label{it-just-the-change-of-the-confidence-level-based-on-the-professors-suggestion.}}

\hypertarget{ii.-alpha}{%
\subparagraph{\texorpdfstring{ii.
\emph{alpha}}{ii. alpha}}\label{ii.-alpha}}

\emph{Explains:}

\hypertarget{from-0.05-to-0.1}{%
\subsection{From 0.05 to 0.1}\label{from-0.05-to-0.1}}

\hypertarget{iii.-increase-the-power-to-reject-the-null-hypothesis.}{%
\subparagraph{\texorpdfstring{iii. \emph{Increase the power to reject
the null
hypothesis.}}{iii. Increase the power to reject the null hypothesis.}}\label{iii.-increase-the-power-to-reject-the-null-hypothesis.}}

\emph{Explains:}

\hypertarget{because-the-critical-point-of-the-right-tail-shifted-leftward.}{%
\subsection{Because the critical point of the right tail shifted
leftward.}\label{because-the-critical-point-of-the-right-tail-shifted-leftward.}}

\hypertarget{iv.-type-i}{%
\subparagraph{\texorpdfstring{iv. \emph{Type
I}}{iv. Type I}}\label{iv.-type-i}}

\emph{Explains:} Type 1 errors has a probability of ``α'' correlated to
the level of confidence we set. Originally, 95\% confidence level means
that there is a 5\% chance of getting a type I error. According to the
suggestion, we set the confidence level to 90\% , the chance of getting
a type 1 error increase (10\%).

\begin{center}\rule{0.5\linewidth}{0.5pt}\end{center}

\hypertarget{d.-your-colleague-has-measured-usage-times-on-five-weekdays-and-taken-a-daily-average.-but-you-feel-this-will-underreport-usage-for-younger-people-who-are-very-active-on-weekends-whereas-it-over-reports-usage-of-older-users.}{%
\paragraph{d.~Your colleague has measured usage times on five weekdays
and taken a daily average. But you feel this will underreport usage for
younger people who are very active on weekends, whereas it over-reports
usage of older
users.}\label{d.-your-colleague-has-measured-usage-times-on-five-weekdays-and-taken-a-daily-average.-but-you-feel-this-will-underreport-usage-for-younger-people-who-are-very-active-on-weekends-whereas-it-over-reports-usage-of-older-users.}}

\begin{center}\rule{0.5\linewidth}{0.5pt}\end{center}

\hypertarget{i.-systematic-error-1}{%
\subsection{\texorpdfstring{\#\#\#\#\# i. \emph{systematic
error}}{\#\#\#\#\# i. systematic error}}\label{i.-systematic-error-1}}

\hypertarget{ii.diff-and-mean-will-be-affected.}{%
\subparagraph{\texorpdfstring{ii.\emph{diff and mean will be
affected.}}{ii.diff and mean will be affected.}}\label{ii.diff-and-mean-will-be-affected.}}

\emph{Explains:} Because the population means we want to inference
includes the user behavior from Monday to Sunday. If we only choose the
workday data as our sample, we will overemphasize the behavior of elder
users and neglect the younger people who are very active on weekends.
Therefore, the mean sample mean is underestimated. According to the
formula of diff(mean - mu 0), as the mean is affected and the mu0 is
still, the diff would be affected for sure.
----------------------------------

\hypertarget{iii.-it-decrease-our-power-to-reject-the-null-hypothesis.}{%
\subparagraph{\texorpdfstring{iii. \emph{it decrease our power to reject
the null
hypothesis.}}{iii. it decrease our power to reject the null hypothesis.}}\label{iii.-it-decrease-our-power-to-reject-the-null-hypothesis.}}

\emph{Reason:} \emph{Explains:} H0 :the mean usage time of the new
smartwatch is the same or less than for the previous smartwatch.

H alt :The mean usage time is greater than that of our previous
smartwatch.

And this error results in underestimating frequent users among the
young. Therefore, it makes us harder to reject H0, which decreases the
power to reject the null hypothesis.

\hypertarget{iv.-type-ii}{%
\subparagraph{\texorpdfstring{iv. \emph{Type
II}}{iv. Type II}}\label{iv.-type-ii}}

\emph{Explains:} Because we should count weekend users in, but we don't.
And false negative means that we should consider specific targets is
negative, but it doesn't.

\begin{center}\rule{0.5\linewidth}{0.5pt}\end{center}

\emph{Explains:}

\hypertarget{section}{%
\subparagraph{}\label{section}}

\begin{Shaded}
\begin{Highlighting}[]
\NormalTok{verizon }\OtherTok{\textless{}{-}} \FunctionTok{read.csv}\NormalTok{(}\StringTok{"verizon.csv"}\NormalTok{)}
\NormalTok{time }\OtherTok{\textless{}{-}}\NormalTok{ verizon}\SpecialCharTok{$}\NormalTok{Time}
\end{Highlighting}
\end{Shaded}

\emph{Ans:}

\begin{center}\rule{0.5\linewidth}{0.5pt}\end{center}

\hypertarget{section-1}{%
\paragraph{}\label{section-1}}

\emph{Ans:}

\begin{center}\rule{0.5\linewidth}{0.5pt}\end{center}

\hypertarget{section-2}{%
\paragraph{}\label{section-2}}

\emph{Ans:}

\begin{center}\rule{0.5\linewidth}{0.5pt}\end{center}

\hypertarget{section-3}{%
\subparagraph{}\label{section-3}}

\emph{Ans:}

\begin{center}\rule{0.5\linewidth}{0.5pt}\end{center}

\begin{center}\rule{0.5\linewidth}{0.5pt}\end{center}

\hypertarget{section-4}{%
\paragraph{}\label{section-4}}

\begin{center}\rule{0.5\linewidth}{0.5pt}\end{center}

\hypertarget{section-5}{%
\subparagraph{}\label{section-5}}

\begin{center}\rule{0.5\linewidth}{0.5pt}\end{center}

\hypertarget{section-6}{%
\subparagraph{}\label{section-6}}

\emph{Ans}:

\begin{center}\rule{0.5\linewidth}{0.5pt}\end{center}

\hypertarget{section-7}{%
\subparagraph{}\label{section-7}}

\emph{Ans: }

\begin{center}\rule{0.5\linewidth}{0.5pt}\end{center}

\hypertarget{section-8}{%
\subparagraph{}\label{section-8}}

\emph{Explain:}

\begin{center}\rule{0.5\linewidth}{0.5pt}\end{center}

\begin{center}\rule{0.5\linewidth}{0.5pt}\end{center}

\hypertarget{section-9}{%
\subparagraph{}\label{section-9}}

\emph{Ans. }

\end{document}
